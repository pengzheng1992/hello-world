\section{Enforce DIAL into SDN}\label{sec:imp}
In this section, we highlight some key designs of enforcing DIAL into real SDN environment.

\textbf{Tag packets.}
To avoid redundant counting in the switches along the routing path of a specific flow, the first switch that counts the flow should tag the packet as ``counted''.
In OpenFlow, we can tag packets using an unused field of packet, \eg, VXLAN.
Such field will be initialized with 0, denoting ``uncounted''.
All the counting rules should only match the uncounted packets.
In real time matching, if a packet is matched with a counting rule, the rule should modify the field to be 1, \ie, the packet has been counted.

\textbf{Update results.}
In the measurement system, the controller will collect the counters of all switches periodically.
To be specific, at the end of each measurement period, switches will report their counter values to the controller by calling \texttt{upload} in Algorithm~\ref{alg:handle}.
This can be implemented by Statistics messages in OpenFlow for querying the counters.

\textbf{Handle exceptions.}
In real-time counting, if a counter exceeds, \texttt{overflow\_report} in Algorithm~\ref{alg:handle} will be triggered to set the switch to be ``FULL''.
If it doesn't exceed, this packet of the flow will remain ``uncounted'' and pass the switches which contain no rule or have been ``FULL'' for the flow.
If all switches of a flow are full or have installed the counting rule, then \texttt{upload} will be triggered to upload the statistics of the flow in all switches to the controller counters and then clear the counter values in the switches to continue counting.
During the counting process of DIAL, when a counter exceeds, the duplication happens as \texttt{overflow\_report} is triggered.
In this procedure, an integer is used to help index the switch which incurs the overflowing.
Then the flow will set the switch ``FULL'' and call \texttt{place} to find another switch to count it on.

In OpenFlow, \texttt{overflow\_report} and \texttt{upload} launched by switch-end can be implemented with Flow-monitor, Experimenter, and Packet\_in messages in special cases.

\begin{algorithm}[t]\small
\caption{Exception handling}
\label{alg:handle}
\KwIn{\KwOut{the flow to be handled: $f$}}
\BlankLine
\texttt{upload}($f$)
\Begin{
	\ForEach {$s$ along the routing path of $f$} {
		\If {the rule of counting $f$ has been installed in $s$}
		{
            upload the counter value of $f$ in $s$ to the controller and update the corresponding controller counter\;
            reset the counter value of $f$ in $s$ to zero\;
            set the state of $s$ along the routing path of $f$ to ``COUNTING''\;
		}
	}
}
\BlankLine
\texttt{overflow\_report}($f$, $i$)
\Begin{
    \If {a packet leads to an overflow} {
        set the state of the $i$-th switch along the routing path of $f$ to ``FULL''\;
		\texttt{place}($f$)\;
    }
}
\end{algorithm}
\vspace{-2ex}
